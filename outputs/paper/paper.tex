% Options for packages loaded elsewhere
\PassOptionsToPackage{unicode}{hyperref}
\PassOptionsToPackage{hyphens}{url}
%
\documentclass[
]{article}
\usepackage{lmodern}
\usepackage{amssymb,amsmath}
\usepackage{ifxetex,ifluatex}
\ifnum 0\ifxetex 1\fi\ifluatex 1\fi=0 % if pdftex
  \usepackage[T1]{fontenc}
  \usepackage[utf8]{inputenc}
  \usepackage{textcomp} % provide euro and other symbols
\else % if luatex or xetex
  \usepackage{unicode-math}
  \defaultfontfeatures{Scale=MatchLowercase}
  \defaultfontfeatures[\rmfamily]{Ligatures=TeX,Scale=1}
  \setmainfont[]{NanumGothic}
\fi
% Use upquote if available, for straight quotes in verbatim environments
\IfFileExists{upquote.sty}{\usepackage{upquote}}{}
\IfFileExists{microtype.sty}{% use microtype if available
  \usepackage[]{microtype}
  \UseMicrotypeSet[protrusion]{basicmath} % disable protrusion for tt fonts
}{}
\makeatletter
\@ifundefined{KOMAClassName}{% if non-KOMA class
  \IfFileExists{parskip.sty}{%
    \usepackage{parskip}
  }{% else
    \setlength{\parindent}{0pt}
    \setlength{\parskip}{6pt plus 2pt minus 1pt}}
}{% if KOMA class
  \KOMAoptions{parskip=half}}
\makeatother
\usepackage{xcolor}
\IfFileExists{xurl.sty}{\usepackage{xurl}}{} % add URL line breaks if available
\IfFileExists{bookmark.sty}{\usepackage{bookmark}}{\usepackage{hyperref}}
\hypersetup{
  pdftitle={Estimating R Anxiety Level Distribution Among Students At the University Using MRP},
  pdfauthor={Yiqu Ding},
  hidelinks,
  pdfcreator={LaTeX via pandoc}}
\urlstyle{same} % disable monospaced font for URLs
\usepackage[margin=1in]{geometry}
\usepackage{graphicx,grffile}
\makeatletter
\def\maxwidth{\ifdim\Gin@nat@width>\linewidth\linewidth\else\Gin@nat@width\fi}
\def\maxheight{\ifdim\Gin@nat@height>\textheight\textheight\else\Gin@nat@height\fi}
\makeatother
% Scale images if necessary, so that they will not overflow the page
% margins by default, and it is still possible to overwrite the defaults
% using explicit options in \includegraphics[width, height, ...]{}
\setkeys{Gin}{width=\maxwidth,height=\maxheight,keepaspectratio}
% Set default figure placement to htbp
\makeatletter
\def\fps@figure{htbp}
\makeatother
\setlength{\emergencystretch}{3em} % prevent overfull lines
\providecommand{\tightlist}{%
  \setlength{\itemsep}{0pt}\setlength{\parskip}{0pt}}
\setcounter{secnumdepth}{-\maxdimen} % remove section numbering

\title{Estimating R Anxiety Level Distribution Among Students At the University
Using MRP\thanks{Code and data are available at:
\url{https://github.com/dding33/STA304-PS5}.}}
\author{Yiqu Ding}
\date{2020-12-16}

\begin{document}
\maketitle
\begin{abstract}
In this report, I look at different factors that affect a studebt's
anxiety level towards using R. The data is obtained from onlien survey
responses which contained variables that could influence anxiety levels.
I then run a multilevel regression on the sample and post-stratify them
using a simulated student census. After the post-stratification, we see
the estimated mean anxiety scores are higher average anxiety scores from
the raw data.

\textbf{Keywords}: MRP, R, Psychology, Education
\end{abstract}

\hypertarget{introduction}{%
\section{Introduction}\label{introduction}}

Relatively speaking, the science of statistics is a new discipline. In
1998, the public image of statistics was poor, and almost nobody knows
what statisticians do{[}from stats{]}. Now, statistics is an essential
tool for nearly all millennial industries. Accompanied by technological
improvements in computers, R has become a necessary tool for all
statistical practitioners; that makes the teaching and training of R
extremely important.

With that being said, R's mastering has not been on student's to-do list
until recent years. At the University of Toronto, up until fall in 2018,
R's learning is not compulsory until third-year courses. Many students
expressed surprise when they first see R's use in the classroom and are
confused. The anxiety issue persists three years after the department
made STA130 compulsory, which was an introduction to statistics and R.
Studies show that the anxiety level affects students' performance in the
classroom and has the potential for further
investigation{[}onlineanxiety{]}.

I am interested in the distribution of R anxiety levels among students.
In this report, I will show the method to estimate the anxiety
distribution among students using a sample I collected from the
University of Toronto. I run a multilevel regression on my sample and
then post-stratify the results using a simulated student census to get
population estimates.

The results of the analysis can be useful in many ways. The university
can periodically conduct this analysis to keep track of teaching
results; generally, this method should be solid for similar reports in
any other university. Students can use the information as a threshold to
understand where they stand among their peers. It is also possible to
study the effect of a treatment such as data camp using this approach,
which potentially saves cost for the department.

In paper we run our analysis in \texttt{R} {[}citeR{]}. We also use the
\texttt{tidyverse} package which was written by tydr.

\hypertarget{data}{%
\section{Data}\label{data}}

Multilevel regression and post-stratification require two data sets. We
train a multilevel regression model using our sample data set(which is
the smaller one), then apply the results to the second data set(usually
a large data set like census) to mimic our population's behavior. In the
context of this report, we want to estimate the anxiety score for all
third-year statistics students at the University of Toronto.

The idea came from an example in mathexample studying anxiety level
towards mathematics.

\hypertarget{sample-from-survey}{%
\subsection{Sample from Survey}\label{sample-from-survey}}

We use piazza and Quercus to distribute the survey organized on Google
Forms\footnote{The link to the full survey:
  \url{https://forms.gle/x4mxCLw6Hh8ecqmT7}} and to record the results.
Naturally, the sampling frame comprises all third-year stats students
who have access to the internet. The survey contains the following
compulsory questions: -one qualification question to reduce sampling
errors; -four demographic questions to post-stratify; -one question
about anxiety level. We ask respondents to self-evaluate their anxiety
level on a scale of 10 when asked to complete a task in R independently,
where `1' represents not very anxious, and `10' represents feeling very
anxious. We will refer to this by `anxiety score' or `anxiety level' for
the rest of this report. In the end, we have an optional question where
the respondent can express their opinion on how to reduce their anxiety
towards R. For privacy reasons, the responses to this question are
masked. Instead, there will be a summary of the responses later in the
discussion section.

We are restricting the year of study because we want to see how the
anxiety levels vary within a group of students with similar exposure to
R; students in the same year have similar experiences both timewise and
course-wise. It is intuitive that the more experience a student has(the
closer he/she is towards graduation), the more familiar he/she becomes
with R and thus has a lower anxiety score.

The total sample size is 48, from which 7 respondents answered `no' to
the qualification questions. That makes the sample size 41. Figure x
shows the distribution of the programs from the sample. The coloring at
the end of the bars indicates the respondent does not have any coding
experience. We notice that this is a small part of the sample.

Figure x displays the anxiety score distribution from the raw data set.
We see two prominent peaks in the distribution: around 3, which
indicates the respondents do not feel very anxious, and around 7.5
indicates the respondents feel quite anxious. Most responses fall
between these two peaks, with few respondents(3 out of 41) reports
extreme anxiety scores towards 1 or 10. Looking at Figure x, we notice
that our sample does not contain any students with a cumulative GPA
lower than C. This skewness means that our sample is biased;
specifically, students with higher cgpa have a stronger incentive to
answer the survey. We will adjust for this in the model by incorporating
random effects.

We must point out that studies show the response biases for sensitive
topics center are near zero, but the responses are unreliable or
noisy{[}sensitive\_info{]}. Since the cgpa and the anxiety score reveal
information about students' academic behavior, we consider them
sensitive topics. We follow steps from {[}mathexample{]} using the
\texttt{brms} package{[}brms1, brms2{]} to adjust for this. Further
explanation will continue in the Model section.

\begin{figure}
\centering
\includegraphics{paper_files/figure-latex/sample-program-1.pdf}
\caption{Distribution of Programs}
\end{figure}

\begin{figure}
\centering
\includegraphics{paper_files/figure-latex/sample-cgpa-1.pdf}
\caption{cgpa Distribution among Sample}
\end{figure}

\begin{figure}
\centering
\includegraphics{paper_files/figure-latex/sample-score-1.pdf}
\caption{Distribution of R Anxiety Scores}
\end{figure}

\hypertarget{simulated-student-census}{%
\subsection{Simulated Student Census}\label{simulated-student-census}}

We simulate a census data set for all third-year stats students and use
this as our post-stratification data. From admission information in
2017(the year that most third-year students in 2020 were admitted), we
estimate our census's size to be 850. It contains five variables that
describe each individual:

\begin{itemize}
\tightlist
\item
  Student\_id;
\item
  sex(2 levels);
\item
  program(8 levels);
\item
  cgpa(5 levels);
\item
  conding\_exp(2 levels).
\end{itemize}

We use ladder four to create post-stratification cells. They contain
information about an individual to divide them into groups and identify
each individual using these four variables. The four variables make
2\emph{8}5*2 = 160 possible cells. The distribution of each variable is
simulated based on a rough estimation of the population. See fig x and x
for a summary of the census. This report's results are not estimates of
the University of Toronto's actual R anxiety distribution, even though
the sample is collected from real respondents.

Based on the census, we developed a few prop data frames for
post-stratification. We counted the number of individuals in each cell
and saved it as cell\_counts.

\begin{figure}
\centering
\includegraphics{paper_files/figure-latex/unnamed-chunk-2-1.pdf}
\caption{Program Distribution in Census}
\end{figure}

\begin{figure}
\centering
\includegraphics{paper_files/figure-latex/unnamed-chunk-4-1.pdf}
\caption{Distribtion of Other Variables in Census}
\end{figure}

\hypertarget{model}{%
\section{Model}\label{model}}

\hypertarget{mrp}{%
\subsection{MRP}\label{mrp}}

We use MRP to predict the anxiety level distribution among our
population. This method adjusts our estimation results by first fitting
a multilevel regression model using the sample, then applying it to the
post-strat data set to predict the population. Specifically, each
individual is defined by his/her sex, program, cgpa, and whether he/she
has previous experience with coding. For each individual in the census,
we predict that person's anxiety score using the previous model using
\texttt{add\_predict\_draws()} from \texttt{tidybayes.} Then we
aggregate the cell-level estimates up to the population level. Using y
to represent the anxiety score,

\begin{equation}
\hat{y}_S^{PS} = \frac{\sum{N_j\hat{y_j}}}{\sum{N_j}}  (\#eq:PS)
\end{equation}

We get our post-stratification estimates by equation (1). You can see
that the key to an accurate estimate relies not only on how well the
model fits the data but also on the level to which the census represents
the population.

We get our post-stratification estimates by equation (1). You can see
that the key to an accurate estimate relies not only on how well the
model fits the data but also on the level to which the census represents
the population. There is often a trade-off between the cells' division
and the prediction results' stability {[}forcast{]}. In our case, the
160 possible cells divide the population very finely(5.3 persons in each
cell on average), which is another reason for us to use MRP. Equation
(2) shows the formula we use for the model, where \(\beta_{pro}\)
represent the coefficient for the program beta, and \(d_{sex}\)
represents the indicating variable for sex.

\begin{equation}
\hat{y} = \beta_0 + \beta_{sex}d_{sex} + \beta_{cgpa}x_{cgpa} + \beta_{pro}x_{pro} + \beta_{code}x_{code} + e (\#eq:formula)
\end{equation}

\hypertarget{model-validation}{%
\subsection{Model Validation}\label{model-validation}}

We perform k-fold cross-validation on the model fitted. This means
refitting the model K times, leaving out one-kth of the original data
each time. We are doing a 3-fold validation because our sample size is
relatively small(41), and dividing it more than three times will lead to
volatile results. We suggest you increase k as the sample size
increases. The cross-validation estimates an average prediction error of
0.003, which indicates the model performance is not problematic.

\begin{verbatim}
## [1] 0.07024705
\end{verbatim}

\hypertarget{results}{%
\section{Results}\label{results}}

We use \texttt{add\_predicted\_draws()} to come up with estimations and
their 95\% confidence intervals. Figures 4-7 show the prediction results
by different sex, program, cgpa, and coding experience, with the raw
data results. We see that the MRP estimates produce a higher mean
anxiety score comparing to the raw results. Specifically,

\begin{itemize}
\tightlist
\item
  MRP estimates substantially different mean anxiety scores for males
  and females, while there was no sign of this pattern among our sample.
  The estimated average for females is 6 and is 5 for the male. There is
  no significant difference between the interval of the groups;
\item
  There is no significant difference in estimated mean anxiety scores
  between different programs. The confidence interval for the Actuarial
  Scitiest is longer than those of other programs, which means a broader
  range of anxiety levels within the program;
\item
  Without any sample, the MRP predicts 5 to be the mean anxiety score
  for students with a cumulative GPA F. This group also has the widest
  confidence interval. Students with a cumulative GPA B seems more
  anxious towards the use of R than any other grades group, with the
  highest mean anxiety score and the narrowest confidence interval; its
  upper boundary is very close to 10, the highest possible anxiety
  score;
\item
  Students with some previous coding experience are estimated to have a
  lower mean anxiety score than those who are new to programming. The
  experienced group also has a much smaller confidence interval, which
  indicates more stability. The inexperienced group's upper boundary
  almost reaches 10, but its lower boundary is close to the lower
  boundary for the experienced group.
\end{itemize}

\begin{figure}
\centering
\includegraphics{paper_files/figure-latex/unnamed-chunk-7-1.pdf}
\caption{MRP estimates vs Raw data in different sex groups}
\end{figure}

\begin{figure}
\centering
\includegraphics{paper_files/figure-latex/unnamed-chunk-8-1.pdf}
\caption{MRP estimates vs Raw data in different programs}
\end{figure}

\begin{figure}
\centering
\includegraphics{paper_files/figure-latex/unnamed-chunk-9-1.pdf}
\caption{MRP estimates vs Raw data within different cgpa groups}
\end{figure}

\begin{figure}
\centering
\includegraphics{paper_files/figure-latex/unnamed-chunk-10-1.pdf}
\caption{MRP estimates vs Raw data with different coding experiences}
\end{figure}

\hypertarget{discussion}{%
\section{Discussion}\label{discussion}}

\hypertarget{limitation-and-future-researches}{%
\subsection{Limitation and Future
Researches}\label{limitation-and-future-researches}}

{[}intro{]} states that MRP works well adjusting for biased samples only
if the under/over-represented variables are present in the
post-stratification data set. With that being said, to get a more
reliable prediction result, future analysis can contain a pre-analysis
which has more variables and use stepwise regression to select variables
that contribute to the accuracy of the model.

\newpage

\hypertarget{appendix}{%
\section*{Appendix}\label{appendix}}
\addcontentsline{toc}{section}{Appendix}

\newpage

\hypertarget{references}{%
\section{References}\label{references}}

\end{document}
